% style
% style


\documentclass[a4paper, 12pt]{article}
%%%%%%%%%%%%%%%%%%%%%%%%%%%%%%%%%%%%%%%%%%%%%%%%%%%%%%%%%%%%%%%%%%%%%%%%%%%%%%%%%%%%%%%%%%%%%%%%%%%%%%%%%%%%%%%%%%%%%%%%%%%%%%%%%%%%%%%%%%%%%%%%%%%%%%%%%%%%%%%%%%%%%%%%%%%%%%%%%%%%%%%%%%%%%%%%%%%%%%%%%%%%%%%%%%%%%%%%%%%%%%%%%%%%%%%%%%%%%%%%%%%%%%%%%%%%
\usepackage{graphicx}
\usepackage{amsmath, amsthm, amssymb}
\usepackage{setspace}
\usepackage{indentfirst}
\usepackage{vmargin}
\usepackage{multirow}
\usepackage{natbib}
\usepackage{tabularx}
\usepackage{url}
\usepackage{bm}
\usepackage[top=1in, bottom=1in, left=1in, right=1in]{geometry}
\usepackage{endnotes}
\usepackage{epsfig}
\usepackage{psfrag}
\usepackage{amsfonts}
\usepackage[T1]{fontenc}
\usepackage{color}
\usepackage{rotating}
\usepackage{longtable}
\usepackage{graphics}
\usepackage{morefloats}
\usepackage{mathrsfs}
\usepackage{subfig}

\setcounter{MaxMatrixCols}{10}
%TCIDATA{OutputFilter=LATEX.DLL}
%TCIDATA{Version=5.50.0.2960}
%TCIDATA{<META NAME="SaveForMode" CONTENT="1">}
%TCIDATA{BibliographyScheme=BibTeX}
%TCIDATA{LastRevised=Wednesday, May 17, 2017 09:30:33}
%TCIDATA{<META NAME="GraphicsSave" CONTENT="32">}
%TCIDATA{Language=American English}

\newcommand {\dsum}{\displaystyle \sum}
\newcommand {\dprod}{\displaystyle \prod}
\setlength{\LTcapwidth}{5in}
\def\p3s{\phantom{xxx}}
\setpapersize{USletter}
\setcounter{secnumdepth}{-2}
\makeatletter
\renewcommand{\section}{\@startsection
{section}    {1}    {0mm}    {-0.7\baselineskip}    {0.08\baselineskip}    {\normalfont\large\sc\center\bf}}
\renewcommand{\subsection}{\@startsection
{subsection}    {2}    {0mm}    {-0.5\baselineskip}    {0.01\baselineskip}    {\normalfont\normalsize\itshape\center}}
\makeatother
\setmarginsrb{1.0in}{1.0in}{1.0in}{0.40in}{0in}{0in}{0in}{0.6in}
\input{tcilatex}
\begin{document}

\date{\today }
\title{Bayesian Inference on \\
Parametric Zombie Survival Model}
\author{Bomin Kim}
\maketitle

\begin{abstract}
\noindent In this paper, we implement Bayesian Inference on the new split
population survival model, that explicitly models the misclassification
probability of failure (vs. right censored) events. This includes two
parametric survival models (Exponential and Weibull) and (possibly) Cox
proportional hazards regression model.
\end{abstract}

\setstretch{1.5}

\clearpage\pagebreak \renewcommand{\thefigure}{\arabic{figure}} %
\setcounter{figure}{0} \renewcommand{\thepage}{\arabic{page}} %
\setcounter{page}{1} \pagestyle{plain} \doublespacing

\section{Review on Parametric Zombie Survival Model}

\subsection{Likelihood function}

\noindent Recall from Ben's "Parametric Zombie Survival Model" that the
probability of misclassification (that is, subset of non-censured failure
outcomes that are being misclassified) is%
\begin{equation}
\alpha =\Pr (C_{i}=1|\widetilde{C}_{i}=0).
\end{equation}%
The unconditional density is thus given by the combination of an
observation's misclassification probability and its probability of
experiencing an actual failure conditional on not being misclassified,%
\begin{equation}
\alpha _{i}+(1-\alpha _{i})\ast f(t_{i})
\end{equation}%
And the unconditional survival function is therefore%
\begin{equation}
(1-\alpha _{i})\ast S(t_{i}),
\end{equation}%
where 
\begin{equation}
\alpha _{i}=\frac{\exp (\mathbf{Z}\gamma )}{1+\exp (\mathbf{Z}\gamma )}.
\end{equation}%
The likeihood function of the Parametric Zombie Survival Model is defined as 
\begin{equation}
L=\displaystyle\prod\limits_{i=1}^{N}[\alpha _{i}+(1-\alpha _{i})f(t_{i}|%
\mathbf{X},\mathbf{\beta })]^{C_{i}}[(1-\alpha _{i})S(t_{i}|\mathbf{X,}%
\mathbf{\beta })]^{1-C_{i}}
\end{equation}%
And the log likelihood is%
\begin{equation}
lnL=\displaystyle\sum\limits_{i=1}^{N}\{C_{i}\ln [\alpha _{i}+(1-\alpha
_{i})f(t_{i}|\mathbf{X},\mathbf{\beta })]+(1-C_{i})\ln [(1-\alpha
_{i})S(t_{i}|\mathbf{X,}\mathbf{\beta })]\}.
\end{equation}%
\iffalse Equivalently, if we define $\delta =1-\alpha $ and substitute this
quantity into Equation (6), the log-likelihood would be defined as: 
\begin{equation}
lnL=\displaystyle\sum\limits_{i=1}^{N}\{C_{i}\ln [(1-\delta _{i})+\delta
_{i}f(t_{i}|\mathbf{X},\mathbf{\beta })]+(1-C_{i})\ln [\delta _{i}S(t_{i}|%
\mathbf{X,}\mathbf{\beta })]\},
\end{equation}%
which is symmetric to the log likelihood of the cure (i.e.,
split-population) survival model. \fi We extend these definitions and
notation from the Parametric Zombie Survival Model to the Cox PH framework
in order to develop the Cox Overreported Failure (OF) model.

\section{Cox-OF Model}

\subsection{Likelihood function}

\noindent The survival function of the conventional Cox Proportional Hazard
(PH) model is%
\begin{equation}
S(t_{i}|\mathbf{X}_{i}\mathbf{,}\beta )=\exp \left\{ -\exp (\mathbf{X}_{i}%
\mathbf{,}\beta )\dint\nolimits_{0}^{t_{i}}h_{0}(u)du\right\}
\end{equation}%
where $\dint\nolimits_{0}^{t_{i}}h_{0}(u)du$ is the cumulative baseline
hazard. The density function of the Cox Proportional Hazard (PH) model is%
\begin{equation}
f(t_{i}|\mathbf{X}_{i}\mathbf{,}\beta )=h(t_{i}|\mathbf{X}_{i})S(t_{i}|%
\mathbf{X}_{i}\mathbf{,}\beta )
\end{equation}%
which means that

\begin{equation}
f(t_{i}|\mathbf{X}_{i}\mathbf{,}\beta )=h_{0}(t_{i})e^{\mathbf{X}_{i}\mathbf{%
,}\beta }\exp \left\{ -\exp (\mathbf{X}_{i}\mathbf{,}\beta
)\dint\nolimits_{0}^{t_{i}}h_{0}(u)du\right\} 
\end{equation}%
where $h(t_{i}|\mathbf{X}_{i})$ is the conditional hazard function and $%
h_{0}(t_{i})$ is the unspecified baseline hazard. Hence the likelihood of
the Cox-OF model is 
\begin{equation}
L=\displaystyle\prod\limits_{i=1}^{N}\left[ \alpha _{i}+(1-\alpha
_{i})h_{0}(t_{i})e^{\mathbf{X}_{i}\mathbf{,}\beta }\exp \left\{ -\exp (%
\mathbf{X}_{i}\mathbf{,}\beta )\dint\nolimits_{0}^{t_{i}}h_{0}(u)du\right\} %
\right] ^{C_{i}}\left[ (1-\alpha _{i})\exp \left\{ -\exp (\mathbf{X}_{i}%
\mathbf{,}\beta )\dint\nolimits_{0}^{t_{i}}h_{0}(u)du\right\} \right]
^{1-C_{i}}
\end{equation}%
\newline
The log-likelihood from this expression is%
\begin{equation}
\ln L=\displaystyle\sum\limits_{i=1}^{N}\left\{ C_{i}\ln \left[ \alpha
_{i}+(1-\alpha _{i})h_{0}(t_{i})e^{\mathbf{X}_{i}\mathbf{,}\beta }\exp
\left( -\exp (\mathbf{X}_{i}\mathbf{,}\beta
)\dint\nolimits_{0}^{t_{i}}h_{0}(u)du\right) \right] +(1-C_{i})\ln \left[
(1-\alpha _{i})\exp \left( -\exp (\mathbf{X}_{i}\mathbf{,}\beta
)\dint\nolimits_{0}^{t_{i}}h_{0}(u)du\right) \right] \right\} 
\end{equation}%
\newline

\subsection{\protect\bigskip E-M estimation of Cox-OF model}

We need some additional notation to describe the E-M
(Expectation-Maximization) algorithm for estimating the Cox-OF model. To
this end, let the log of the marginal full likleihood for the Cox-OF model
be defined (more briefly) as%
\begin{equation}
l_{f}(\mathbf{\theta },\Lambda _{0})=\displaystyle\sum\limits_{i=1}^{N}%
\{C_{i}\ln [\alpha _{i}+(1-\alpha _{i})h(t_{i}|\mathbf{X}_{i})S(t_{i}|%
\mathbf{X}_{i}\mathbf{,}\beta )]+(1-C_{i})\ln [(1-\alpha _{i})S(t_{i}|%
\mathbf{X}_{i}\mathbf{,}\beta )]\}
\end{equation}%
where $h(t_{i}|\mathbf{X}_{i})=h_{0}(t_{i})e^{\mathbf{X}_{i}\mathbf{,}\beta
} $and $S(t_{i}|\mathbf{X}_{i}\mathbf{,}\beta )=\exp \left\{ -\exp (\mathbf{X%
}_{i}\mathbf{,}\beta )\Lambda _{0}(t)\right\} $ with $\Lambda _{0}(t)=$ $%
\dint\nolimits_{0}^{t_{i}}h_{0}(u)du$. Let $\mathbf{\theta }=(\mathbf{\gamma 
},\mathbf{\beta })$. The estimate of $(\mathbf{\theta },\Lambda _{0})$ is
obtained by maximizing $l_{f}$ over $(\mathbf{\theta },\Lambda _{0})$. This
maximization is performed using the E-M algorithm given a suitable starting
value $(\mathbf{\theta }^{(0)},\Lambda _{0}^{(0)})$ of $(\mathbf{\theta }%
,\Lambda _{0})$. Once we have $\mathbf{\theta }^{(0)}$, a suitable $\Lambda
_{0}^{(0)}$ corresponding to $\mathbf{\theta }^{(0)}$ can be computed by
using the Newton-Raphson method. If we have appropriate starting values for $%
(\mathbf{\theta }^{(0)},\Lambda _{0}^{(0)})$, then -- following Taylor
(1995) and Sy and Taylor (2000) who developed the E-M algorithm for the
classic Cox cure model -- we can formally describe the E-M algorithm for the
Cox-OF model as follows. To start, the \textit{m-}th E-step in the E-M
algorithm first transforms $l_{f}$ into the following form for the
complete-data log likelihood%
\begin{equation}
l_{f}^{EM}(\mathbf{\gamma |w}^{(m)})=\sum\limits_{i=1}^{N}\left[
(1-w_{i}^{(m)}(t_{i}))\ln \alpha _{i}+w_{i}^{(m)}(t_{i})\ln (1-\alpha _{i})%
\right]
\end{equation}%
\begin{equation}
l_{f}^{EM}(\mathbf{\beta ,}\Lambda _{0}\mathbf{|w}^{(m)})=\sum%
\limits_{i=1}^{N}\left[ C_{i}\ln h_{0}(t_{i})+C_{i}\mathbf{X}_{i}\mathbf{%
\beta }+w_{i}^{(m)}(t_{i})\ln f(t_{i})\right]
\end{equation}%
for a given $\mathbf{w}^{(m)}=(w_{1}^{(m)}(t_{1}),...,w_{N}^{(m)}(t_{N}))(m%
\geq 1),$ where $\mathbf{w}^{(m)}$ is computed by current parameter values $(%
\mathbf{\theta }^{(m-1)},\Lambda _{0}^{(m-1)})$ and%
\begin{equation}
w_{i}(t_{i};\mathbf{\theta },\Lambda _{0})=\left\{ 
\begin{array}{c}
\frac{(1-\alpha _{i})f(t_{i})}{\alpha _{i}+(1-\alpha _{i})f(t_{i})}\text{ \
\ \ \ }C_{i}=0 \\ 
1\text{ \ \ \ \ \ \ \ \ \ \ \ \ \ \ \ \ \ \ \ \ \ }C_{i}=1%
\end{array}%
\right\}
\end{equation}%
In the \textit{m-}th M-step in the E-M algorithm, the Breslow's estimate of $%
\Lambda _{0}$,%
\begin{equation}
\widehat{\Lambda }(t_{i}|\mathbf{\beta ,w}^{(m)})=\displaystyle %
\sum\limits_{\{i;T_{i}\leq t\}}\frac{C_{i}}{\tsum\limits_{j=1}^{N}I(T_{j}%
\geq T_{i})\exp (\mathbf{X}_{j}\mathbf{\beta })w_{j}^{(m)}(T_{j})}
\end{equation}%
(where $I$ is an indicator function) can be employed for the maximization of 
$l_{f}^{EM}(\mathbf{\beta ,}\Lambda _{0}\mathbf{|w}^{(m)})$. Substituting $%
\widehat{\Lambda }_{0}(t_{i}|\mathbf{\beta ,w}^{(m)})$ into $\Lambda _{0}$
of $l_{f}^{EM}(\mathbf{\beta ,}\Lambda _{0}\mathbf{|w}^{(m)})$ leads to the
following log partial likelihood%
\begin{equation}
l_{p}^{EM}(\mathbf{\beta |w}^{(m)})=\sum\limits_{i=1}^{N}C_{i}\left[ \mathbf{%
X}_{i}\mathbf{\beta }-\ln \left( \tsum\limits_{j=1}^{N}I(T_{j}\geq
T_{i})\exp (\mathbf{X}_{j}\mathbf{\beta })w_{j}^{(m)}(T_{j})\right) \right]
\end{equation}%
The M-step of $l_{f}^{EM}(\mathbf{\beta ,}\Lambda _{0}\mathbf{|w}^{(m)})$
over $(\mathbf{\beta ,}\Lambda _{0})$ is then replaced by maximizing $%
l_{p}^{EM}(\mathbf{\beta |w}^{(m)})$ only over $\mathbf{\beta }$ and by
treating $\Lambda _{0}$ as a nonparametric function. Hence, the M-step is
easily performed by maximizing $l_{f}^{EM}(\mathbf{\gamma |w}^{(m)})$ and $%
l_{p}^{EM}(\mathbf{\beta |w}^{(m)})$ over $\mathbf{\gamma }$ and $\mathbf{%
\beta }$ via the Newton-Raphson method.

Now suppose that $\mathbf{\theta }^{(m)}=(\mathbf{\gamma }^{(m)},\mathbf{%
\beta }^{(m)})$ are the current estimates obtained by the M-step for a given 
$\mathbf{w}^{(m)}$. This means that the current estimate $\Lambda _{0}^{(m)}$%
of $\Lambda _{0}$ can be written as $\widehat{\Lambda }_{0}(.|\mathbf{\beta }%
^{(m)}\mathbf{,w}^{(m)})$. Further, for the next $(m+1)$-th E-step, $\mathbf{%
w}^{(m)}$ is updated to $\mathbf{w}^{(m+1)}$ by substituting $(\mathbf{%
\theta }^{(m)},$ $\Lambda _{0}^{(m)})$ into $(\mathbf{\theta },$ $\Lambda
_{0})$ in $\widehat{\Lambda }(t_{i}|\mathbf{\beta ,w}^{(m)})$. Finally, $(%
\mathbf{\theta }^{(\widehat{m})},$ $\Lambda _{0}^{(\widehat{m})})$ which
provides $\widehat{m}=\arg \max_{m=0,1...}.l_{f}(\mathbf{\theta }^{(m)},$ $%
\Lambda _{0}^{(m)})$ becomes an estimate of $(\mathbf{\theta },$ $\Lambda
_{0})$ in a series of the E-M iteration.

\subsection{\protect\bigskip Posterior Distribution of Cox-OF model}

For the Cox-OF model, we need to assign a prior to each of the following
parameters: the baseline hazard $h_{0}$ which is labeled as $h$ for
convenience, $\mathbf{\beta }=\{\beta _{1},...,\beta _{p_{1}}\}$, and $%
\mathbf{\gamma }=\{\gamma _{1},...,\gamma _{p_{2}}\}$. For the baseline
hazard, we follow standard practice\footnote{%
complete footnote} and assume a Gamma prior where both parameters are set
equal to $0.001$.\footnote{%
This implies that the expectation is equal to 1 and the variance is equal to
1000.} Hence%
\begin{equation}
h\symbol{126}\text{ Gamma}(a_{h},b_{h})
\end{equation}%
where $a_{h}=b_{h}=0.001$. From the parametric Zombie survival models, we
assume the prior of $\mathbf{\beta }=\{\beta _{1},...,\beta _{p_{1}}\}$ as 
\begin{equation}
\mathbf{\beta }\sim \mbox{MVN}_{p_{1}}(\mathbf{\mu }_{\beta },\Sigma _{\beta
}),
\end{equation}%
Thus the conditional posterior distribution for $h_{0}$ and $\mathbf{\beta }$
is respectively given by%
\begin{equation}
\pi (h|\mathbf{C},\mathbf{\alpha },\mathbf{X},\mathbf{Z},\mathbf{t},\mathbf{%
\beta ,\gamma })\propto L(h|\mathbf{C},\mathbf{\alpha },\mathbf{X},\mathbf{Z}%
,\mathbf{t},\mathbf{\beta ,\gamma })\times \pi (h|a_{h},b_{h})
\end{equation}%
\begin{equation}
\pi (\mathbf{\beta }|\mathbf{C},\mathbf{\alpha },\mathbf{X},\mathbf{Z},%
\mathbf{t},\mathbf{\gamma })\propto L(\mathbf{\beta }|\mathbf{C},\mathbf{%
\alpha },\mathbf{X},\mathbf{Z},\mathbf{t},\mathbf{\beta ,\gamma })\times \pi
(\mathbf{\beta }|\mathbf{\mu }_{\beta },\Sigma _{\beta })
\end{equation}%
Further, following the parametric Zombie survival models, we can assign
mutivariate Normal prior to $\mathbf{\gamma }=\{\gamma _{1},...,\gamma
_{p_{2}}\}$, 
\begin{equation}
\mathbf{\gamma }\sim \mbox{MVN}_{p_{2}}(\mathbf{\mu }_{\gamma },\Sigma
_{\gamma }),
\end{equation}%
and the corresponding conditional posterior distribution of $\mathbf{\gamma }
$ becomes 
\begin{equation}
\pi (\mathbf{\gamma }|\mathbf{C},\mathbf{X},\mathbf{Z},\mathbf{t},\mathbf{%
\beta ,h})\propto L(\mathbf{\gamma }|\mathbf{C},\mathbf{X},\mathbf{Z},%
\mathbf{t},\mathbf{\beta ,h})\times \pi (\mathbf{\gamma }|\mathbf{\mu }%
_{\gamma },\Sigma _{\gamma }).
\end{equation}

\textit{What about the cumulative basline hazard? }

\subsection{MCMC algorithm for Cox-OF\ model}

\noindent

\end{document}
